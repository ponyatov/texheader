% Universal LaTeX headers for e-book publications
\documentclass[oneside,12pt]{book}
%% mobile phone optimized
\usepackage[paperwidth=210mm,paperheight=145mm,margin=2mm]{geometry}
%% font setup for screen reading
\renewcommand{\familydefault}{\sfdefault}\normalfont
%% hyperlinks pdf style
\usepackage[unicode,colorlinks=true]{hyperref}

% xcolor fixes
\usepackage{xcolor}
\definecolor{red}{rgb}{0.7,0,0}		% R
\definecolor{RED}{rgb}{0.7,0,0}
\definecolor{green}{rgb}{0,0.7,0}	% G
\definecolor{GREEN}{rgb}{0,0.7,0}
\definecolor{blue}{rgb}{0,0,0.7}	% B
\definecolor{BLUE}{rgb}{0,0,0.7}

% Cyrillization
%% \usepackage[T1,T2A]{fontenc}
\usepackage[utf8]{inputenc}
%% \usepackage[cp1251]{inputenc}
\usepackage[english,russian]{babel}
\usepackage{indentfirst}

% relative sectioning
\usepackage{ifthen}
\newcounter{secdepth}\setcounter{secdepth}{0}
\newcommand{\secup}{\addtocounter{secdepth}{1}}
\newcommand{\secdown}{\addtocounter{secdepth}{-1}}
\newcommand{\secrel}[1]{
\ifthenelse{\equal{\value{secdepth}}{0}}{\part{#1}}{}
\ifthenelse{\equal{\value{secdepth}}{-1}}{\chapter{#1}}{}
\ifthenelse{\equal{\value{secdepth}}{-2}}{\section{#1}}{}
\ifthenelse{\equal{\value{secdepth}}{-3}}{\subsection{#1}}{}
\ifthenelse{\equal{\value{secdepth}}{-4}}{\subsubsection{#1}}{}
}
\newcommand{\secly}[1]{
\section*{#1}
\addcontentsline{toc}{section}{#1}
}

% computer-related markup: manuals, listings, CS

%% software menu & keys
\usepackage[os=win]{menukeys}

%% listings
\usepackage{verbatim}
\usepackage{listings}
\lstset{
basicstyle=\small,
frame=single,
numbers=left,numberstyle=\small,numbersep=2mm,
xleftmargin=3mm,xrightmargin=2mm,
tabsize=4,
keywordstyle=\textbf,
commentstyle=\color{blue}\textbf,
extendedchars=\true
}
\newcommand{\lst}[1]{\lstinputlisting{#1}}

%% languages
\newcommand{\ci}{Си}
\newcommand{\cpp}{$C^+_+$}
\newcommand{\py}{$Python$}
\newcommand{\java}{$Java$}
\newcommand{\racket}{$Racket$}

% misc
\newcommand{\email}[1]{$<$\href{mailto:#1}{#1}$>$}
\renewcommand{\emph}[1]{\textcolor{red}{#1}}
\newcommand{\term}[1]{\textcolor{green}{#1}}
\newcommand{\termdef}[2]{\textcolor{green}{#1}\index{#2}}
\newcommand{\note}[1]{\,\footnote{\ #1}}
\newcommand{\ru}[1]{\textcolor{blue}{#1}}